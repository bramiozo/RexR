For the cohort-bias removal we apply a genome-wise Location and scale (L/S) adjustment per cohort. This guarantees
similar bounds and means over the different cohorts. To remove the multiplicity of probesets per patients
we simply take the mean of the measurements per patient. Having obtained a normalised dataset we add $5\%$
of noise (with respect the maximum value) to increase the robustness of the final model. To further increase robustness, and improve 
interpretability we reduce the number of features by applying the FDR method with ANOVA for the distribution comparison, 
following the Benjamin-Hochberg procedure, using a maximum $p$-value of $0.05$. 
%
We have two data scenario's. Scenario $\mathcal{I}$ uses the data from adult patients with a known high/low intensity diagnosis (where 
medium intensity is considered equal to the low intensity diagnosis) to train a model to predict the diagnosis for the other non-labelled patients.
Scenario $\mathcal{II}$ uses the survival label of the patients as a proxy for the high/low intensity treatment. Now the training set from scenario
$\mathcal{I}$ is used as a validation set.
%
For the model creation we employ several \textit{tree-ensemble methods}: Random Forest (RF) by Breiman\cite{Breiman2001}, ExtraTrees (ET) by Geurts et al.\cite{Geurts2006} 
XGBoost (XGB) by Chen and Guestrin\cite{Chen2016} and (Light)GBM (LGBM) by Ke et al.\cite{Ke2017}. 
We use three types of linear methods, Linear Discrimination Analysis, Logistic Regression and linear SVM.
We use two nonlinear methods, both \textit{neural networks}, a Deep Neural Network (DNN) \cite{lecun2015deep} and a Convolutional Neural Network (CNN) \cite{Lecun98}.
Finally, we use two Bayesian methods, Naive Bayes and Gaussian Processes Classication (GPC).
%
Finally we combine them in a so-called stacked method, either by a majority vote of the classifications or by weighted averaging based on their prediction probabilities.
The benefit of combining these techniques is that we have the highest possible accuracy without compromising interpretability. \\ \\
%
Having obtained the models we can extract their relative importance from the tree-based models, and the weights from the linear models.
In our case we could reduce the more than $50.000$ expression values to about $200$ without losing accuracy. 

- TO DO : build stacker
- TO DO : apply to training scenario II
- reproduce table from scenario $\mathcal{I}$